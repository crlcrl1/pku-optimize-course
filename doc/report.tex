\documentclass{article}

\usepackage{neurips_2022}
\usepackage[utf8]{inputenc}
\usepackage[T1]{fontenc}
\usepackage{hyperref}
\usepackage{url}
\usepackage{booktabs}
\usepackage{amsfonts}
\usepackage{nicefrac}
\usepackage{microtype}
\usepackage{xcolor}
\usepackage{amsmath}
\usepackage{amssymb}
\usepackage[UTF8]{ctex}
\usepackage{algorithm}
\usepackage{algpseudocode}
\usepackage{resizegather}
\usepackage{graphicx}
\usepackage{subfigure}

\hypersetup{
    colorlinks=true,
    linkbordercolor=white
}
\setlength{\parindent}{2em}

\title{最优化方法上机作业报告}

\author{陈润璘 2200010848}

\begin{document}

\maketitle

\section{问题描述}

本次上机作业使用多种不同的方法求解 Group LASSO 问题:
\begin{equation*}
    \min_{x \in \mathbb{R}^{n\times l}} \frac{1}{2} \|Ax - b\|_2^2 +
    \mu \|x\|_{1,2}
\end{equation*}
其中 $A \in \mathbb{R}^{m\times n}$, $b \in \mathbb{R}^{m\times l}$,
\begin{equation*}
    \|x\|_{1,2} = \sum_{i=1}^n \|x(i, 1:l)\|_2
\end{equation*}
其中 $x(i, 1:l)$ 表示矩阵 $x$ 的第 $i$ 行.

\section{算法实现}

\subsection{使用 CVX 直接求解}

Python 中的 \href{https://www.cvxpy.org/}{CVXPY} 软件包可以接受多种类型的优化问题作为输入, 并且支持多种优化问题求解器作为后端. 它自动将输入的优化问题自动转换为后端求解器可以接受的形式, 调用求解器进行求解.

在本次作业中, 我们使用 CVXPY 分别调用 Mosek 和 Gurobi 求解器求解 Group LASSO 问题. 使用这两个求解器的获得的结果将作为一些评价指标的基准.

\subsection{使用 Mosek 和 Gurobi 求解}

\href{https://www.mosek.com/}{Mosek} 和 \href{https://www.gurobi.com/}{Gurobi} 是两个商业的优化求解器, 他们都提供了 Python 接口. 在本次作业中, 我们使用这两个求解器求解 Group LASSO 问题.

\subsubsection{Mosek}

Mosek 求解器可以求解标准形式的二次锥规划问题, 因此我们需要将 Group LASSO 问题转换为二次锥规划问题. 我们可以将 Group LASSO 问题转换为如下 SOCP 的形式:
\begin{equation*}
    \begin{aligned}
        \min_{s,t}\quad & \frac{1}{2} t + \mu (s_1 + s_2 + \cdots + s_n)\\
        \text{s.t.}\quad & \|Ax - b\|_F^2 \le 2 \cdot t \cdot t_0\\
        &t_0 = \frac{1}{2}\\
        &\|x(i, :)\|_2 \le s_i,\,, i = 1, 2, \cdots, n
    \end{aligned}
\end{equation*}

在文件 \verb|gl_mosek.py| 中实现了这个转换, 并使用 Mosek 求解器求解.

\subsubsection{Gurobi}
由于 Gurobi 的 Python 接口可以方便地添加二次约束, 因此在使用 Gurobi 求解器进行求解, 我们把 Group LASSO 问题转换为一个二次规划问题:
\begin{equation*}
    \begin{aligned}
        \min_{x, z}\quad & \frac{1}{2} t + \mu (s_1 + s_2 + ... + s_n)\\
        \text{s.t.}\quad &\|Ax-b\|_F^2 \le t^2\\
        &\|x(i, :)\|_2^2 \le s_i^2,\,, i = 1, 2, \cdots, n\\
        &t \ge 0, s_i \ge 0,\,, i = 1, 2, \cdots, n
    \end{aligned}
\end{equation*}

在文件 \verb|gl_gurobi.py| 中实现了这个转换, 并使用 Gurobi 求解器求解.

\subsection{次梯度法和光滑化的梯度法}

\subsubsection{次梯度法}
由于问题中的目标函数并不是处处可微的, 因此智能使用次梯度法进行求解. 不难计算出目标函数关于 $x$ 的次梯度
\begin{equation*}
    \partial f(x) = A^T(Ax - b) + \mu \partial \|x\|_{1,2}
\end{equation*}
其中 $\partial \|x\|_{1,2}$ 表示 $x$ 的次梯度, 其第 $i$ 行为
\begin{equation*}
    \partial \|x\|_{1,2}(i, :) =
    \begin{cases}
        \frac{x(i, :)}{\|x(i, :)\|_2} & x(i, :) \neq 0\\
        \text{任意向量} & x(i, :) = 0
    \end{cases}
\end{equation*}
在实际实现中, 对于 $x(i, :) = 0$ 的情况, 我们可以取 $\partial \|x\|_{1,2}(i, :)=0$.

\subsubsection{光滑化的梯度法}
为了使用梯度法求解问题, 我们可以对目标函数进行光滑化近似, 使其处处可微. 为此, 将目标函数中的 $\|x\|_{1,2}$ 替换为其光滑化近似函数:
\begin{equation*}
    l_\sigma(x) =
    \begin{cases}
        \|x\|_2^2 / 2\sigma, &\text{if } |x| \leq \sigma, \\
        \|x\|_2 - \sigma / 2, &\text{otherwise},
    \end{cases}
\end{equation*}
其中 $\sigma > 0$ 是一个参数, 并随着迭代步数的增加逐渐减小. 对于 $l_\sigma(x)$, 我们可以计算其梯度
\begin{equation*}
    \nabla l_\sigma(x) =
    \begin{cases}
        x / \sigma, &\text{if } |x| \leq \sigma, \\
        x / \|x\|_2, &\text{otherwise},
    \end{cases}
\end{equation*}

\subsubsection{连续化策略}
在实际实现中, 不难发现当目标函数中的 $\mu$ 较小时, 无论使用次梯度法还是光滑化后的梯度法, 目标函数的值都很难收敛到最优值. 为了解决这个问题, 我们可以使用连续化策略, 将目标函数中的 $\mu$ 替换为一个较大的值, 并在迭代过程中逐渐减小 $\mu$ 的值直到它等于原来的值. 由于 $\mu$ 越大时这个优化问题越容易求解, 因此 这个方法大大加快了收敛速度.

\subsubsection{步长选取}
在算法实现中不难发现使用消失步长会因为迭代次数的增加导致步长过小, 从而使得算法收敛速度变慢, 而使用固定步长则对导致最优值在迭代后期无法进一步下降. 因此, 我们使用了固定步长和消失步长混合的策略. 具体地, 我们使用如下的步长选取策略:
\begin{equation*}
    \alpha_k =
    \begin{cases}
        \alpha_0, & \mu > \mu_0\, \text{or}\, (\mu=\mu_0\, \text{and}\, k \le k_0)\\
        \frac{\alpha_0}{\sqrt{k-k_0}}, & \mu = \mu_0\,\text{and}\, k > k_0\\
    \end{cases}
\end{equation*}
其中 $\alpha_0 = \frac{1}{\|A\|_2^2}$, $\mu_0$ 是 $\mu$ 的原始值, $k$ 是使用 $\mu_0$ 作为参数时的迭代次数, $k_0$ 是一个参数.

\subsubsection{算法框架}
总结上述的讨论, 我们可以得到如下的算法框架:
\begin{algorithm}
    \caption{次梯度算法或光滑化的梯度法}
    \begin{algorithmic}
        \State $x \gets x_0$, list $\gets [\mu\cdot t^s, \mu\cdot t^{s-1}, \cdots \mu \cdot t, \mu]$ \Comment{生成一个递减的 $\mu$ 序列}
        \State $\sigma = \sigma_0$ \Comment{如果使用了光滑化的算法, 初始化 $\sigma$}
        \For{$\mu$ in list}
        \State $k \gets 0$
        \While{not converge and $k < k_{\max}$}
        \State 计算步长 $t$
        \State 计算(次)梯度 grad
        \State $x \gets x - t \cdot \text{grad}$ \Comment{更新 $x$}
        \State $k \gets k + 1$
        \EndWhile
        \State $\sigma \gets \sigma / \beta$ \Comment{如果使用了光滑化的算法, 更新 $\sigma$}
        \EndFor
    \end{algorithmic}
    \label{alg:gd}
\end{algorithm}
\newline
其中, 只有在使用了光滑化的算法时才需要初始化或更新 $\sigma$, 其他情况下不需要.

\subsection{近似点梯度法和 Nesterov 加速}

\subsubsection{近似点梯度法}
由于问题中的目标函数由一个可微函数和一个不可微函数组成, 因此我们可以使用近似点梯度法进行求解. 近似点梯度法的基本思想是在每次迭代中, 先使用梯度法更新可微的部分, 再应用\textbf{邻近算子}更新不可微的部分. 具体地, 在本问题中, 我们可以使用如下的迭代公式:
\begin{equation*}
    x_{k+1} = \text{prox}_{t_k\mu\|\cdot\|_{1,2}}(x_k - t_k \cdot \nabla f(x_k))
\end{equation*}
其中用邻近算子 $\text{prox}_h(x)$ 的定义如下:
\begin{equation*}
    \text{prox}_h(x) = \arg\min_y \left\{\frac{1}{2}\|y - x\|_2^2 + h(y)
    \right\}
\end{equation*}
在这个问题中, 我们可以计算 $\text{prox}_{t\|\cdot\|_{1,2}}(x)$ 的解析解:
\begin{equation*}
    \text{prox}_{t\|\cdot\|_{1,2}}(x)(i,1:l) =
    \begin{cases}
        \left(1-\frac{t}{\|x(i,1:l)\|_2}\right)x, & \|x(i,1:l)\|_2 \ge t\\
        0, & \|x(i,1:l)\|_2 < t
    \end{cases}
\end{equation*}

近似点梯度法的框架与算法 \ref{alg:gd} 类似, 只是在更新 $x$ 时使用了邻近算子代替了次梯度.

\subsubsection{Nesterov 加速}
对于优化问题
\begin{equation*}
    \min_x\quad f(x)=g(x)+h(x)
\end{equation*}
其中 $g(x)$ 是可微凸函数, $h(x)$ 是不可微闭凸函数, 可以在近似点梯度法的基础上使用 Nesterov 算法加速. Nesterov 算法的迭代公式如下:
\begin{equation*}
    \begin{aligned}
        y &= x_{k-1} + \frac{k-2}{k+1}(x_{k-1} - x_{k-2})\\
        x_k &= \text{prox}_{t_k h}(y - t_k \nabla g(y))
    \end{aligned}
\end{equation*}
在上式中我们约定 $x_{-1} = x_0$. 无论从理论还是算法实现的角度, Nesterov 算法都可以加速收敛速度.

\subsection{增广拉格朗日函数法}

在这一部分中, 我们能对 Group LASSO 的对偶问题使用增广拉格朗日函数法进行求解. Group LASSO 的对偶问题为
\begin{equation*}
    \begin{aligned}
        \min_{Y\in \mathbb{R}^{m\times l}}\quad &\left<b, Y\right> + \frac{1}{2} \|Y\|_F^2
        \\
        \text{s.t.}\quad & \|A^T Y\|_{\infty,2} \leq \mu
    \end{aligned}
\end{equation*}
为了使用增广拉格朗日函数法求解这个问题, 我们把对偶问题改写为如下的形式:
\begin{equation}
    \begin{aligned}
        \min_{Y\in \mathbb{R}^{m\times l}}\quad &\left<b, Y\right> + \frac{1}{2} \|Y\|_F^2
        \\
        \text{s.t.}\quad & A^T Y = Z\\
        & \|Z\|_{\infty,2} \leq \mu
    \end{aligned}
    \label{eq:dual}
\end{equation}
然后, 构造如下的增广拉格朗日函数:
\begin{equation*}
    L_\sigma(Y,Z;X) = \left<b, Y\right> + \frac{1}{2} \|Y\|_F^2 + \left<X, A^T Y - Z\right> + \frac{\sigma}{2} \|A^T Y - Z\|_F^2,\quad \|Z\|_{\infty,2} \leq \mu
\end{equation*}

对于增广拉格朗日函数, 使用如下的迭代格式:

\begin{algorithm}
    \caption{增广拉格朗日函数法}
    \begin{algorithmic}
        \State $Y \gets Y_0$, $Z \gets Z_0$, $X \gets X_0$ \Comment 初始化变量和乘子
        \While{$\|A^T Y - Z\|_F > tol$}
        \State $Y, Z \gets \arg \min_{Y,Z} L_\sigma(Y,Z;X)$ \Comment 更新变量
        \State $X \gets X + \sigma(A^T Y -Z)$ \Comment 更新乘子
        \EndWhile
    \end{algorithmic}
    \Return $-X$ \Comment 在构造的增广拉格朗日函数中, $X$ 与原问题的解差一个符号
\end{algorithm}

\subsection{交替方向乘子法}

交替方法乘子法是一种求高效解优化问题的方法, 它通过将原问题分解为若干个子问题, 并通过交替求解这些子问题来求解原问题. 具体地, 对于优化问题
\begin{equation*}
    \begin{aligned}
        \min_{x_1, x_2}\quad & f_1(x_1) + f_2(x_2)\\
        \text{s.t.}\quad & A_1 x_1 + A_2 x_2 = b
    \end{aligned}
\end{equation*}
构造增广拉格朗日函数:
\begin{equation*}
    L_\rho(x_1, x_2; \lambda) = f_1(x_1) + f_2(x_2) + \left<\lambda, A_1 x_1 + A_2 x_2 - b\right> + \frac{\rho}{2}\|A_1 x_1 + A_2 x_2 - b\|_2^2
\end{equation*}
其中 $\lambda$ 是拉格朗日乘子, $\rho$ 是惩罚因子. 交替方向乘子法的迭代格式如下:
\begin{algorithm}
    \caption{交替方向乘子法}
    \begin{algorithmic}
        \State $x_1 \gets x_{1,0}$, $x_2 \gets x_{2,0}$, $\lambda \gets \lambda_0$ \Comment 初始化变量和乘子
        \While{not converge}
        \State $x_1 \gets \arg \min_{x_1} L_\rho(x_1,x_2;\lambda)$ \Comment 更新 $x_1$
        \State $x_2 \gets \arg \min_{x_2} L_\rho(x_1,x_2;\lambda)
        $ \Comment 更新 $x_2$
        \State $\lambda \gets \lambda + \tau(A_1 x_1 + A_2 x_2 - b)$ \Comment 更新乘子
        \EndWhile
    \end{algorithmic}
\end{algorithm}
\newline
其中 $\tau\in \left(0,\frac{1+\sqrt{5}}{2}\right]$ 是步长.

对于 Group LASSO 的对偶问题, 同样将其按公式 \ref{eq:dual} 拆分可以使用交替方向乘子法求解. 对于原问题, 可以类似地拆分:
\begin{equation*}
    \begin{aligned}
        \min_{x \in \mathbb{R}^{n\times l}}\quad &\frac{1}{2} \|Ax - b\|_2^2 +
        \mu \|y\|_{1,2}\\
        \text{s.t.}\quad & x=y
    \end{aligned}
\end{equation*}
然后构造增广拉格朗日函数:
\begin{equation*}
    L_\rho(x, y; \lambda) = \frac{1}{2} \|Ax - b\|_2^2 + \mu \|y\|_{1,2} + \left<\lambda, x - y\right> + \frac{\rho}{2}\|x - y\|_2^2
\end{equation*}
使用交替方向乘子法求解. 同时, 我们可以对在对 $y$ 求极小值时使用线性化策略, 及在每次迭代时, 用
\begin{equation*}
    y_{k+1} = \arg\min_y \frac{1}{2} \mu \|y\|_{1,2} + \left<\lambda_k, x_k - y\right> + \frac{1}{2\alpha}\|x_k - y\|_2^2
\end{equation*}
来更新 $y$, 这相当于做一步近似点梯度下降.

\section{数值结果}

\subsection{评价指标}
我们使用多种指标来评价不同算法的性能:
\begin{itemize}
    \item 目标函数值 (Fval): 目标函数值越小, 说明算法越优秀.
    \item 算法求得的解与 Mosek 求得的解的误差 (ErrMosek): 使用公式 $\frac{\|x - x_{\text{mosek}}\|_F}{1+\|x_{\text{mosek}}\|_F}$ 计算.
    \item 算法求得的解与 Gurobi 求得的解的误差 (ErrGurobi): 使用公式 $\frac{\|x - x_{\text{gurobi}}\|_F}{1+\|x_{\text{gurobi}}\|_F}$ 计算.
    \item 算法求得的的解与真解的误差 (ErrExact): 使用公式 $\frac{\|x - x_{\text{exact}}\|_F}{1+\|x_{\text{exact}}\|_F}$ 计算.
    \item 解的稀疏度 (Sparsity): 在这个问题中, 解的稀疏度定义为解中绝对值大于所有元素绝对值最大者的 $1e-6$ 倍的元素的个数除以总元素个数.
    \item 算法运行时间 (Time): 先进行 50 次循环预热, 然后进行 500 次循环, 计算平均时间.
    \item 算法迭代次数 (Iter): 算法迭代的次数, 用于评价算法的收敛速度.
\end{itemize}

\subsection{测试环境}

\begin{itemize}
    \item OS: ArchLinux 6.12.4-arch1-1
    \item CPU: Intel(R) Core(TM) i7-13700H CPU @ 5.0GHz
    \item Python: 3.12.7
    \item NumPy: 2.2.0 compiled with MKL and TBB (\href{https://aur.archlinux.org/packages/python-numpy-mkl-tbb}{AUR Package})
    \item Mosek: 10.1.29
    \item Gurobi: 12.0.0
\end{itemize}

\subsection{测试结果及分析}
各算法的测试结果如表 \ref{tab:result} 所示.
\begin{table}[h]
    \centering
    \resizebox{\textwidth}{!}{
        \begin{tabular}{c|ccccccc}
            \toprule
            \hline
            Algorithm & Fval & ErrMosek & ErrGurobi & ErrExact & Sparsity & Time(ms) & Iter\\
            \hline
            CVX-Mosek & 0.67057523 & 0.000000e+00 & 2.616487e-07 & 3.186258e-05 & 0.1025 & 272.3105 & 9\\
            CVX-Gurobi & 0.67057524 & 2.616487e-07 & 0.000000e+00 & 3.199129e-05 & 0.1045 & 523.2296 & 10\\
            Mosek & 0.67057522 & 1.838438e-07 & 4.217172e-07 & 3.177695e-05 & 0.1025 & 180.4748 & 10\\
            Gurobi & 0.67058679 & 3.174155e-05 & 3.186995e-05 & 1.748776e-07 & 0.0996 & 1072.6585 & 8\\
            Subgradient & 0.67057529 & 2.903549e-06 & 9.105488e-06 & 2.409748e-06 & 0.0996 & 210.9301 & 1887\\
            Smoothed-Gradient & 0.67057529 & 3.042466e-06 & 2.592209e-06 & 3.340779e-05 & 0.0996 & 110.7308 & 856\\
            Proximate-Gradient & 0.67057529 & 2.993171e-06 & 2.620100e-06 & 3.331597e-05 & 0.0996 & 67.9360 & 684\\
            Fast-Proximate-Gradient & 0.67057561 & 1.998454e-06 & 7.089518e-06 & 9.049563e-06 & 0.0996 & 25.5381 & 198\\
            Augmented-Lagrangian-Dual & 0.67058902 & 3.235087e-05 & 3.226961e-05 & 5.904511e-05 & 0.0996 & 20.7258 & 255\\
            Alternating-Direction-Dual & 0.67058799 & 3.082449e-05 & 3.072649e-05 & 5.906829e-05 & 0.0996 & 8.0282 & 66\\
            Alternating-Direction-Primal & 0.67057523 & 2.294772e-07 & 4.606795e-07 & 3.176232e-05 & 0.1025 & 73.6252 & 1375\\
            \hline
            \bottomrule
        \end{tabular}
    }
    \caption{不同算法的测试结果}
    \label{tab:result}
\end{table}

\subsubsection{次梯度法和光滑化的梯度法}

从结果中可以看出, 即使使用了连续化策略, 次梯度法和光滑化的梯度法这两种直接的梯度算法的表现都不如其他算法. 此外, 在解的稀疏性上, 这两种算法也不如其他的算法. 从迭代次数和运行时间上看, 使用光滑的函数近似不可微的 $\|\cdot\|_{1,2}$ 可以提升算法的收敛速度.

图 \ref{fig:gd} 展示了次梯度法和光滑化的梯度法的收敛曲线. 从图中可以看出, 使用了连续化策略后, 两种算法的目标函数值和梯度的范数都呈阶梯状下降, 在每次重启后都会有一个较大的下降, 从而加快了收敛速度.

\begin{figure}[h]
    \centering
    \subfigure[次梯度法]{
        \includegraphics[width=0.455\textwidth]{img/SGD.png}
    }
    \subfigure[光滑化的梯度法]{
        \includegraphics[width=0.448\textwidth]{img/GD.png}
    }
    \caption{次梯度法和光滑化的梯度法的收敛曲线}
    \label{fig:gd}
\end{figure}

\subsubsection{近似点梯度法和 Nesterov 加速}

由于使用了邻近算子处理不可微的部分, 近似点梯度法的表现要好于次梯度法和光滑化的梯度法, 并且其求解速度快于两个商业求解器. 使用 Nesterov 加速后, 算法的收敛速度有了明显的提升. 在解的稀疏性上, 由于邻近算子会将范数较小的行直接置为 0, 因此近似点梯度法的解的稀疏性更好, 并且在不同的随机数种子下, 近似点梯度法的解的稀疏性也更加稳定.

图 \ref{fig:prox} 展示了近似点梯度法和 Nesterov 加速的收敛曲线. 与图 \ref{fig:gd} 相比, 在使用了邻近算子后, 梯度不会出现重启后突然下降再快速上升的锯齿状图像, 而是更加平滑地下降.

\begin{figure}[h]
    \centering
    \subfigure[近似点梯度法]{
        \includegraphics[width=0.45\textwidth]{img/prox.png}
    }
    \subfigure[Nesterov 加速的近似点梯度法]{
        \includegraphics[width=0.46\textwidth]{img/fprox.png}
    }
    \caption{近似点梯度法和 Nesterov 加速的近似点梯度法的收敛曲线}
    \label{fig:prox}
\end{figure}

\subsubsection{增广拉格朗日函数法}

对于目标函数的最优值, 对对偶问题使用增广拉格朗日函数法的结果要略差于其他算法, 但是其解的稀疏性要好于其他算法. 从迭代次数和运行时间上看, 增广拉格朗日函数法的表现也要好于次梯度法和光滑化的梯度法.

图 \ref{fig:ALM} 展示了增广拉格朗日函数法的收敛曲线. 从图中可以看出, 在没有使用连续化策略的情况下, 目标函数的值和对偶间隙并没有呈阶梯状下降, 而是先平滑地下降, 然后在一定的迭代次数后稳定.

\begin{figure}[h]
    \centering
    \includegraphics[width=0.5\textwidth]{img/ALM_dual.png}
    \caption{增广拉格朗日函数法的收敛曲线}
    \label{fig:ALM}
\end{figure}

\subsubsection{交替方向乘子法}

对于对偶问题, 交替方向乘子法的表现要好于增广拉格朗日函数法, 并且是所有算法中最快的. 从解的稀疏性上看, 交替方向乘子法的解的稀疏性与增广拉格朗日函数法相当. 从迭代次数和运行时间上看, 交替方向乘子法的表现也要好于增广拉格朗日函数法. 但是, 与增广拉格朗日函数法相同, 交替方向乘子法的最优值要略差于其他算法.

对于原问题, 交替方向乘子法得到的解十分接近 Mosek 和 Gurobi 的解, 并且解的稀疏性也很好. 但是, 直接对原问题使用交替方向乘子法的收敛速度要慢于对偶问题.

图 \ref{fig:ADMM} 展示了交替方向乘子法的收敛曲线. 从图中可以看出, 与增广拉格朗日函数法类似, 对偶问题的交替方向乘子法的目标函数值和对偶间隙也是先平滑地下降, 然后在一定的迭代次数后稳定. 对原问题, 交替方向乘子法的目标函数值先迅速下降, 然后缓慢下降直到稳定.

\begin{figure}[h]
    \centering
    \subfigure[对偶问题的收敛曲线]{
        \includegraphics[width=0.52\textwidth]{img/ADMM_dual.png}
    }
    \subfigure[原问题的收敛曲线]{
        \includegraphics[width=0.25\textwidth]{img/ADMM_primal.png}
    }
    \caption{交替方向乘子法的收敛曲线}
    \label{fig:ADMM}
\end{figure}

\section{总结}

在本次作业中, 我们使用了多种算法求解 Group LASSO 问题, 并对这些算法的性能进行了评价. 从结果中可以看出, 针对问题的结构和性质设计和优化的算法往往比通用的优化算法更加高效. 在实际应用中, 我们可以根据问题的性质选择合适的算法, 从而提高算法的效率.

\end{document}
